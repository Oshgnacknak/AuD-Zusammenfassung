\documentclass[a4paper,12pt]{article}
\usepackage{amssymb} % needed for math
\usepackage{amsmath} % needed for math
\usepackage[utf8]{inputenc} % this is needed for german umlauts
\usepackage[ngerman]{babel} % this is needed for german umlauts
\usepackage[T1]{fontenc}    % this is needed for correct output of umlauts in pdf
\usepackage[margin=2.5cm]{geometry} %layout
\usepackage{hyperref}  % this is needed for forms and links within the text
\usepackage{tikz}
\usepackage{xifthen}
\usepackage{eforms} % <-- the driver is pdftex or xetex
\begin{document} 
\noindent Führen Sie die Unterroutine MERGE in Mergesort auf dem Array
\begin{align*}
    A[54 \ldots 63]=[20,22,52,68,89,14,39,56,75,78]
\end{align*}
aus. Der Algorithmus durchlauft in einer for-Schleife nacheinander die Elemente des Arrays. Geben Sie zuerst die
Tellarrays $L$ und $R$ an und anschlie\ss{}end den Zustand des Array $A$ nach jedem Durchlauf der for-Schleife sowie direkt
vor dem Ende an. Verwenden Sie den Algorithmus, der in der Vorlesung gezeigt wurde. Geben Sie den Wert von $i, j$
und $k$ nach jeder tteration der for-Schleife sowie direkt vor dem Ende von MERGE konkret an.\\
Damit die Überprüfung funktioniert, schreibe "`inf"' (ohne Gänsefü\ss{}chen) für unendlich.
\def\textfieldwidth{.8cm}
\begin{defineJS}{\aNiceScript}
    var isCorrect = true;
    var expectedL = [20,22,52,68,89, "inf"];
    var expectedR = [14,39,56,75,78, "inf"];
    var expected = [
        [14,22,52,68,89,14,39,56,75,78, 0, 1, 54],
        [14,20,52,68,89,14,39,56,75,78, 1, 1, 55],
        [14,20,22,68,89,14,39,56,75,78, 2, 1, 56],
        [14,20,22,39,89,14,39,56,75,78, 2, 2, 57],
        [14,20,22,39,52,14,39,56,75,78, 3, 2, 58],
        [14,20,22,39,52,56,39,56,75,78, 3, 3, 59],
        [14,20,22,39,52,56,68,56,75,78, 4, 3, 60],
        [14,20,22,39,52,56,68,75,75,78, 4, 4, 61],
        [14,20,22,39,52,56,68,75,78,78, 4, 5, 62],
        [14,20,22,39,52,56,68,75,78,89, 5, 5, 63],
        [14,20,22,39,52,56,68,75,78,89]
        ];
    for(var i = 0; i < expectedL.length;i++){
        if(expectedL[i]!=this.getField("l"+(i+1)).value) {
            isCorrect=false;
            app.alert("Fehler in L, Spalte " + i + ": Erwartet wurde <" + expectedL[i] + "> aber gefunden wurde <" + this.getField("l"+(i+1)).value + ">.");
            break;
        }
    }
    if(isCorrect){
        for(var i = 0; i < expectedR.length;i++){
            if(expectedR[i]!=this.getField("r"+(i+1)).value) {
                isCorrect=false;
                app.alert("Fehler in R, Spalte " + i + ": Erwartet wurde <" + expectedR[i] + "> aber gefunden wurde <" + this.getField("r"+(i+1)).value + ">.");
                break;
            }
        }
    }
    if(isCorrect){
        aloop:for(var i = 1; i <= 11; i++) {
            for(var j = 1; j <= 13; j++) {
                if(j<11 && expected[i-1][j-1]!=this.getField("merge"+i+"-"+j).value){
                    isCorrect=false;
                    app.alert("Fehler in Zeile " + i + ", Spalte " + j + ": Erwartet wurde <" + expected[i-1][j-1] + "> aber gefunden wurde <" + this.getField("merge"+i+"-"+j).value  +  ">.");
                    break aloop;
                } else if(i<11) {
                    if(j == 11 && expected[i-1][j-1]!==this.getField("merge"+i+"-i").value){
                        isCorrect=false;
                        app.alert("Fehler in Zeile " + i + ", Spalte i: Erwartet wurde <" + expected[i-1][j-1] + "> aber gefunden wurde <" + this.getField("merge"+i+"-i").value  +  ">."); 
                        break aloop;
                    }
                    if(j == 12 && expected[i-1][j-1]!==this.getField("merge"+i+"-j").value){
                        isCorrect=false;
                        app.alert("Fehler in Zeile " + i + ", Spalte j: Erwartet wurde <" + expected[i-1][j-1] + "> aber gefunden wurde <" + this.getField("merge"+i+"-j").value  +  ">."); 
                        break aloop;
                    }
                    if(j == 13 && expected[i-1][j-1]!==this.getField("merge"+i+"-k").value){
                        isCorrect=false;
                        app.alert("Fehler in Zeile " + i + ", Spalte k: Erwartet wurde <" + expected[i-1][j-1] + "> aber gefunden wurde <" + this.getField("merge"+i+"-k").value  +  ">."); 
                        break aloop;
                    }
                }
            }
        }
    }
    if(isCorrect){
        app.alert("Deine Lösung ist Korrekt.");
    } 
\end{defineJS}
\begin{Form}[action={}]
    [20,22,52,68,89,14,39,56,75,78]\\
    $L[$
                    \TextField[name=l1, width=\textfieldwidth,charsize=12pt, maxlen=3]{}$,$
                    \TextField[name=l2, width=\textfieldwidth,charsize=12pt, maxlen=3]{}$,$
                    \TextField[name=l3, width=\textfieldwidth,charsize=12pt, maxlen=3]{}$,$
                    \TextField[name=l4, width=\textfieldwidth,charsize=12pt, maxlen=3]{}$,$
                    \TextField[name=l5, width=\textfieldwidth,charsize=12pt, maxlen=3]{}$,$
                    \TextField[name=l6, width=\textfieldwidth,charsize=12pt, maxlen=3]{}$]$\\
    $R[$
                    \TextField[name=r1,width=\textfieldwidth,charsize=12pt, maxlen=3]{}$,$
                    \TextField[name=r2,width=\textfieldwidth,charsize=12pt, maxlen=3]{}$,$
                    \TextField[name=r3,width=\textfieldwidth,charsize=12pt, maxlen=3]{}$,$
                    \TextField[name=r4,width=\textfieldwidth,charsize=12pt, maxlen=3]{}$,$
                    \TextField[name=r5,width=\textfieldwidth,charsize=12pt, maxlen=3]{}$,$
                    \TextField[name=r6,width=\textfieldwidth,charsize=12pt, maxlen=3]{}$]$\\
    \linebreak
    \foreach \i in {1,...,11}{
            $[$
                            \TextField[name=merge\i-1, width=\textfieldwidth,charsize=12pt, maxlen=3]{}$,$
                            \TextField[name=merge\i-2, width=\textfieldwidth,charsize=12pt, maxlen=3]{}$,$
                            \TextField[name=merge\i-3, width=\textfieldwidth,charsize=12pt, maxlen=3]{}$,$
                            \TextField[name=merge\i-4, width=\textfieldwidth,charsize=12pt, maxlen=3]{}$,$
                            \TextField[name=merge\i-5, width=\textfieldwidth,charsize=12pt, maxlen=3]{}$,$
                            \TextField[name=merge\i-6, width=\textfieldwidth,charsize=12pt, maxlen=3]{}$,$
                            \TextField[name=merge\i-7, width=\textfieldwidth,charsize=12pt, maxlen=3]{}$,$
                            \TextField[name=merge\i-8, width=\textfieldwidth,charsize=12pt, maxlen=3]{}$,$
                            \TextField[name=merge\i-9, width=\textfieldwidth,charsize=12pt, maxlen=3]{}$,$
                            \TextField[name=merge\i-10,width=\textfieldwidth,charsize=12pt, maxlen=3]{}$]$
            \qquad
            \ifthenelse{\i = 11}{}{
                $i=$\TextField[name=merge\i-i,width=\textfieldwidth,charsize=12pt]{}$,
                    j=$\TextField[name=merge\i-j,width=\textfieldwidth,charsize=12pt]{}$,
                    k=$\TextField[name=merge\i-k,width=\textfieldwidth,charsize=12pt]{}
            }
            \\
        }
    \vspace*{.3cm}
    \phantom{a}\\
    \pushButton[\BC{0 0 1}\BG{0.98 0.92 0.73}
    \textColor{1 0 0}\CA{Prüfen}\AC{Checken}\A{\JS{\aNiceScript}}]
    {myComboButton}{2cm}{1cm}
    \Reset{Reset}
    \hfill ~\\
\end{Form}
\end{document}
