\documentclass[
    ngerman,
    color=3b,
    dark_mode,
    load_common, % Loads a list of commonly used Packages
    summary,
    boxarc,
    % manual_term,
    % solution=true,
]{rubos-tuda-template} 
% Import all Packages from Main Preamble with relative Path
% \subimport*{../../}{preamble}
% Get Labels from Main Document using the xr-hyper Package
\externaldocument{../../AuD-Zusammenfassung-2020}
% Set Graphics Path, so pictures load correctly
\graphicspath{{../../}}


\begin{document}
\section{Advanced Data Structures}\label{Advanced Data Structures}\index{Advanced Data Structures}
\subsection{Rot-Schwarz-Bäume}\label{Rot-Schwarz-Baeume}\index{Rot-Schwarz-Bäume}
\begin{definition}[Rot-Schwarz-Baum]\mbox{}
    \begin{itemize}
        \item Ist ein binärer Suchbaum mitden zusätzlichen Eigenschaften:
              \begin{itemize}
                  \item Jeder Knoten hat die Farbe rot oder schwarz
                  \item Die Wurzel ist schwarz
                  \item Wenn ein Knoten rot ist, sind seine Kinder schwarz ("`Nicht-Rot-Rot-Regel"')
                  \item Für jeden Knoten hat jeder Pfad zu einem Blatt oder Halbblatt die selbe Anzahl an schwarzen Knoten
              \end{itemize}
        \item Halbblätter im RBT sind schwarz
        \item Schwarzhöhe\index{Schwarzhöhe} eines Knoten: \\
              Eindeutige Anzahl von schwarzen Knoten auf dem Weg zu einem Blatt im Teilbaum des Knoten
        \item Für leeren Baum gibt Schwarzhöhe = 0 ($SH(nil)=0$)
    \end{itemize}
\end{definition}
\paragraph{Höhe eines Rot-Schwarz-Baums}\mbox{}
\begin{grayInfoBox}
    \begin{itemize}
        \item $h \leq 2 \cdot log_2(n+1)$  ($n$ Knoten)
        \item In jedem Unterteilbaum gleiche Anzahl schwarzer Knoten
        \item Maximal zusätzlich gleiche Anzahl roter Knoten auf diesem Pfad
        \item Einigerma\ss{}en ausbalanciert $\Rightarrow$ Höhe $O(log~n)$
    \end{itemize}
\end{grayInfoBox}
Alle folgenden Algorithmen arbeiten mithilfe eines Sentinels (zeigt auf sich selbst)

\begin{figure}[ht]
    \centering
    \includestandalone[width=.5\textwidth]{pictures/baum_beispiel/baum_beispiel}% 
    \caption{Beispielhafte Darstellung wie in \ref{Definitionen fuer Datenstrukturen}}
    \label{fig:red_black_tree_example}
\end{figure}
\clearpage
\paragraph{Einfügen}
\begin{enumerate}
    \item Finde Elternknoten wie im BST (\hyperref[BST-Insert]{BST-Einfüge Algorithmus})
    \item Knoten als Kind von Elternknoten anfügen
    \item Färbe den neuen Knoten rot
    \item Wiederherstellen der Rot-Schwarz-Bedingung\index{fixColorsAfterInsertion()}
\end{enumerate}
%\begin{noindent}        
\begin{codeBlock}[autogobble,escapeinside=||]{title={insert(T,z) //z.left==z.right==nil;}}
    x=T.root; px=T.sent;
    WHILE x != nil DO                   //Bis zum passenden Blatt gehen
        px=x;
        IF x.key > z.key THEN
            x=x.left;
        ELSE
            x=x.right;
    z.parent=px;
    IF px==T.sent THEN                  // Einfügen
        T.root=z
    ELSE
        IF px.key > z.key THEN
            px.left=z;
        ELSE
            px.right=z;
    z.color=red;                        // ab hier anders als bei BST-Insert
    fixColorsAfterInsertion(T,z);
\end{codeBlock} 
%\end{noindent}
Laufzeit: $\Theta(h)$ ($h$ jedoch $log~n$)

Hilfsmethode \texttt{rotateLeft}\index{rotateLeft()}
%\begin{noindent}        
\begin{codeBlock}[autogobble]{title={rotateLeft(T,x)}}
    y = x.right;
    x.right = y.left;
    IF y.left != nil THEN
        y.left.parent = x;
    y.parent = x.parent;
    IF x.parent == T.sent THEN
        T.root = y;
    ELSE
        IF x == x.parent.left THEN
            x.parent.left = y;
        ELSE
            x.parent.right = y;
    y.left = x;
    x.parent = y;
\end{codeBlock}
%\end{noindent}
\clearpage
\paragraph{fixColorsAfterInsertion}
Beim Aufrufen werden zwei Bedingungen potentiell verletzt:
\begin{enumerate}
    \item root ist nicht mehr schwarz
    \item wenn eine Node rot ist müssen beide Kinder schwarz sein
\end{enumerate}
Also müssen wir:
\begin{itemize}
    \item nach Rotation die root des Baumes ggf auf schwarz setzen
    \item Überprüfen, ob RSB-Bedingung verletzt wurde
    \item Blo\ss{} wenn das parent auch rot ist kommen wir in Verlegenheit $\Longrightarrow$ wir müssen den Algorithmus starten
\end{itemize}

Fälle falls $z.parent$ rot ist:\\
\begin{minipage}{.5\textwidth-1.85301pt}
    \begin{itemize}
        \item[1.] Onkel ist ebenfalls rot $\Rightarrow$ parent und Onkel werden schwarz gefärbt und Grandparent wird rot gefärbt, z wird zu z.p.p, ggf. Verletzung durch Farbe des neuen z
        \item[2.] Onkel ist schwarz und z hat andere Kindrichtung wie z.p $\Rightarrow$ zu Fall 3 durch Rotation konvertieren
        \item[3.] Onkel ist schwarz und z hat gleiche Kindrichtung wie z.p $\Rightarrow$ z.p wird schwarz, z.p.p wird rot, und es wird um z.p.p entgegen der Kindrichtung gedreht. Schleife terminiert nach Fall 3
    \end{itemize}
\end{minipage}
\begin{minipage}{.5\textwidth}
    \begin{tcolorbox}[
            colback=yellow!20,
            colframe=black!70,
            enhanced,
            title={Fall 1},
            fonttitle=\sffamily\bfseries,
            center title,
            every float=\centering
        ]
        \centering
        \includestandalone{pictures/fixColorsAfterInsertionCases/case1}
    \end{tcolorbox}
\end{minipage}\\
\begin{minipage}[t]{.5\textwidth}\mbox{}
    \begin{tcolorbox}[
            colback=yellow!20,
            colframe=black!70,
            enhanced,
            title={Fall 2},
            fonttitle=\sffamily\bfseries,
            center title,
            every float=\centering
        ]
        \centering
        \includestandalone{pictures/fixColorsAfterInsertionCases/case2}
    \end{tcolorbox}
    %\captionof*{figure}{Fall 2}
\end{minipage}
\begin{minipage}[t]{.5\textwidth-1.85301pt}\mbox{}
    \begin{tcolorbox}[
            colback=yellow!20,
            colframe=black!70,
            enhanced,
            title={Fall 3},
            fonttitle=\sffamily\bfseries,
            center title,
            every float=\centering,
        ]
        \centering
        \includestandalone{pictures/fixColorsAfterInsertionCases/case3}
    \end{tcolorbox}
    %\captionof*{figure}{Fall 3}
\end{minipage}
%\begin{noindent}
        \begin{codeBlock}[autogobble,fontsize=\small]{title={fixColorsAfterInsertion(T,z)}}
            WHILE z.parent.color == red DO                  // solange der Elternknoten rot ist
                IF z.parent == z.parent.parent.left THEN    // Linkes Kind (if-Fall)
                    y = z.parent.parent.right;
                    IF y != nil AND y.color == red THEN     // Fall 1
                        z.parent.color = black;
                        y.color = black;
                        z.parent.parent.color = red;
                        z = z.parent.parent;                // rekursiv nach oben weiterführen
                    ELSE                                    
                        IF z == z.parent.right THEN         // Fall 2
                            z = z.parent;
                            rotateLeft(T,z);
                        z.parent.color = black;             // Fall 3
                        z.parent.parent.color = red;
                        rotateRight(T, z.parent.parent);
                ELSE                                        
                    // Tauschen von rechts und links        
                T.root.color = black;                       // Setzen der Wurzel auf Schwarz
            \end{codeBlock}
            %\end{noindent}
\clearpage
\paragraph{Löschen}
\begin{itemize}
    \item analog zum binären Suchbaum, aber neue Node erbt Farbe der alten Node
    \item Wenn "`neue"' Node schwarz war $\Rightarrow$ Fixup
    \item Verschiedene Fälle, die auch gegenseitig Voraussetzungen füreinander sind
\end{itemize}
Hilfsroutine transplant:\index{transplant}
%\begin{noindent}
\begin{codeBlock}[autogobble,fontsize=\small]{title={transplant(T,u,v)}}
    IF u.parent==nil THEN
        T.root=v;
    ELSE
        IF u==u.parent.left THEN
            u.parent.left=v;
        ELSE
            u.parent.right=v;
    IF v != nil THEN
        v.parent=u.parent;
\end{codeBlock}
%\end{noindent}
%\begin{noindent}
\begin{codeBlock}[autogobble,fontsize=\small]{title={delete(T,z)}}
    y=z;
    y-original-color=y.color;
    IF z.left == nil;
        x = z.right;
        transplant(T,z,z.right);
    ELSE IF z.right == nil;
        x = z.left;
        transplant(T,z,z.left);
    ELSE
        y TREE-MINIMUM(z.right);
        y-original-color=y.color;
        x=y.right;
        IF y.p == z
            x.p = y;
        ELSE
            transplant(T,y,y.right);
            y.right=z.right;
            y.right.p=y;
        transplant(T,z,y);
        y.left=z.left;
        y.left.p=y;
        y.color=z.color;
    IF y-original-color == BLACK
        deleteFixup(T,x);
\end{codeBlock}
%\end{noindent}
Laufzeit: $O(h) = O(log~n)$
\clearpage
Delete kann die RSB-Bedingung verletzen. Das fixup hat vier fälle:
\begin{figure}[h]
    \centering
    \includegraphics[width=\textwidth-2cm]{pictures/deleteFixuoRBTCases\IfDarkModeT{_dark}.png}
    \caption{Fälle für Fixup bei delete in RSB}
\end{figure}
\begin{enumerate}
    \item Knoten $w$ (Bruder von Knoten $x$) ist rot
          \begin{itemize}
              \item Da $w$ schwarze Kinder haben muss, können wir die Farben von $w$ und $x.p$ wechseln und dann eine Linksdrehung auf $x.p$ ausführen, ohne eine der rot-schwarzen Eigenschaften zu verletzen.
              \item Das neue Geschwisterchen von $x$, das vor der Rotation eines der Kinder von $w$ ist, ist jetzt schwarz. Wir haben also Fall $1$ in Fall $2, 3$ oder $4$ konvertiert.
                    Die Fälle $ 2, 3$ und $4$ treten auf, wenn der Knoten $w$ schwarz ist, aber unterschiedliche Farben der Kinder.
          \end{itemize}
    \item $w$ ist schwarz und beide Kinder von $w$ sind schwarz
          \begin{itemize}
              \item entfernen eines Schwarz von $x$ und $w$, wobei $x$ nur ein Schwarz hat und $w$ rot bleibt
              \item Um Entfernen aus $x$ und $w$ zu kompensieren ein zusätzliches Schwarz hinzufügen.
              \item w konnte entweder Schwarz oder rot sein.
              \item Wenn $x.p~(=a)$ schwarz, dann Vaterknoten $\Delta SH=-1$; verfahre rekursiv mit $x.p~(=a)$ als neuem $x$; $\Rightarrow$ wir wiederholen die while-Schleife mit $x.p$ als neuem Knoten $x$.
          \end{itemize}
    \item $w$ ist schwarz, $w$'s linkes Kind ist rot und $w$'s rechtes Kind ist schwarz
          \begin{itemize}
              \item Farben von $w$ und seinem linken Kind links ändern und dann eine Rechtsdrehung auf $w$ ausführen
              \item Das neue Geschwister $w$ von $x$ ist jetzt ein schwarzer Knoten mit einem roten rechten Kind
              \item[$\Rightarrow$] Fall $3$ in Fall $4$ transformiert.
          \end{itemize}
    \item $w$ ist schwarz und das rechte Kind von $w$ ist rot
          \begin{itemize}
              \item $w$ erbt die Farbe von $x.p$
              \item $x.p$ wird schwarz
              \item $w.right$ wird schwarz
              \item[$\Rightarrow$] LinksRotation von $x.p$
              \item $x$ als Wurzel festlegen, wird der Whileloop beendet, wenn die Schleifenbedingung getestet wird.
          \end{itemize}
\end{enumerate}

%\begin{noindent}
                \begin{codeBlock}[autogobble,fontsize=\small]{title={deleteFixup(T,z)}}
                    WHILE x != T.root and x.color == BLACK
                        IF x==x.p.left
                            w=x.p.right;
                            IF w.color == RED                           // Case 1
                                w.color=BLACK;
                                x.p.color=RED;
                                rotateLeft(T,x.p);
                                w=x.p.right;
                            IF w.left.color == w.right.color == BLACK   // Case 2
                                w.color = RED;
                                x=x.p;
                            ELSE IF w.right.color == BLACK              // Case 3
                                w.left.color = BLACK;
                                w.color=RED;
                                rotateRight(T,w);
                                w=x.p.right;
                            w.color=x.p.color;                          // Case 4
                            x.p.color=BLACK;
                            w.right.color=BLACK;
                            rotateLeft(T,x.p);
                            x=T.root;
                        ELSE
                            // same as above with "right" and "left" exchanged
                    x.color=BLACK;
                \end{codeBlock}
                %\end{noindent}

\paragraph{Worst-Case-Laufzeiten}
\begin{description}[leftmargin=2cm]
    \item [Einfügen] $\Theta(log~n)$
    \item [Löschen] $\Theta(log~n)$
    \item [Suchen] $\Theta(log~n)$
\end{description}
\clearpage
\subsection{AVL-Bäume}\label{AVL-Baeume}\index{AVL-Bäume}
\begin{wrapfigure}[1]{r}{.4\textwidth}
    % \centering
    \includegraphics[width=7cm]{pictures/avlbaum\IfDarkModeT{_dark}.PNG}
    \caption{Beispiel Balance AVL Baum}
\end{wrapfigure}\mbox{}
\begin{definition}[AVL-Baum]\mbox{}\\
    Binärer Suchbaum, aber für Balance in \fatsf{jedem} Knoten nur $-1,0,$ oder $1$ erlaubt.\\
    Balance für $x$: $B(x)$ = Höhe(rechter Teilbaum) - Höhe(linker Teilbaum)
    \index{linkslastig}\index{rechtslastig}\index{balanciert}
\end{definition}
$h \leq 1.441 \cdot log~n$ (optimierte Konstanten - 1,441 vs 2 (RBT))\\
\paragraph{AVL vs. Rot-Schwarz-Bäume}\mbox{}
\begin{grayInfoBox}
    \begin{description}[leftmargin=3cm,itemsep=1em]
        \item [AVL]
              \begin{itemize}
                  \item Einfügen und Löschen verletzen in der Regel öfter die Baum-Bedingung
                  \item Aufwendiger zum Rebalancieren
              \end{itemize}
        \item [Rot-Schwarz]
              \begin{itemize}
                  \item Suchen dauert evtl. länger
              \end{itemize}
        \item [Konklusion]
              \begin{itemize}
                  \item AVL geeigneter, wenn mehr Such-Operationen und weniger Einfügen und Löschen
              \end{itemize}
        \item [Gemeinsamkeiten]\begin{itemize}
                  \item AVL $\subset$ Rot-Schwarz
                  \item AVL-Baum $\Rightarrow$ Rot-Schwarz-Baum mit Höhe $\left \lceil \frac{h+1}{2} \right \rceil$
                  \item Für jede Höhe $h \geq 3$ gibt es einen RBT, der kein AVL-Baum ist (AVL $\neq$ RBT)
              \end{itemize}
    \end{description}
\end{grayInfoBox}
\clearpage
\paragraph{Einfügen}
\begin{itemize}
    \item Einfügen funktioniert wie beim Binary Search Tree mit Sentinel\index{Sentinel}
    \item Erfordert danach jedoch Rebalancieren weiter oben im Baum
    \item Rebalancieren: (verschiedene Fälle)
\end{itemize}

\begin{minipage}{0.5\textwidth}
    \centering\includegraphics[width=\textwidth]{pictures/avlcase1\IfDarkModeT{_dark}.PNG}
\end{minipage}
\begin{minipage}{0.5\textwidth}
    \centering\includegraphics[width=\textwidth]{pictures/avlcase34\IfDarkModeT{_dark}.PNG}
\end{minipage}

\begin{minipage}{0.5\textwidth}
    \centering\includegraphics[width=\textwidth]{pictures/avlcase2_1\IfDarkModeT{_dark}.PNG}
\end{minipage}
\begin{minipage}{0.5\textwidth}
    \centering\includegraphics[width=\textwidth]{pictures/avlcase2_2\IfDarkModeT{_dark}.PNG}
\end{minipage}


\paragraph{Löschen}
\begin{itemize}
    \item Analog zum binären Suchbaum
    \item Rebalancieren (eventuell bis zur Wurzel) notwendig
\end{itemize}

\paragraph{Worst-Case-Laufzeiten}
\begin{description}[leftmargin=2.5cm]
    \item [Einfügen] $\Theta(log~n)$
    \item [Löschen] $\Theta(log~n)$
    \item [Suchen] $\Theta(log~n)$
\end{description}
\begin{itemize}
    \item theoretisch bessere Konstanten als RBT
    \item in Praxis aber nur unwesentlich schneller
\end{itemize}
\clearpage
\subsection{Splay-Bäume}\label{Splay-Baeume}\index{Splay-Bäume}
\begin{definition}[Splay-Baum]\mbox{}
    \begin{itemize}
        \item selbst-organisierender Baum
        \item Ansatz: Einmal angefragte Werte werden wahrs. noch öfter angefragt
        \item Angefragte Werte nach oben schieben
        \item Splay-Bäume sind eine Untermenge von Rot-Schwarz-Bäumen ($\subseteq$)
    \end{itemize}
\end{definition}

\paragraph{Splay-Operationen}
\begin{itemize}
    \item Suchen oder Einfügen: Spüle gesuchten oder neu eingefügten Knoten an die Wurzel
    \item Splay:\index{splay} Folge von Zig-,Zig-Zig-, Zig-Zag-Operationen
\end{itemize}
%\begin{noindent}
\begin{codeBlock}[autogobble]{title={splay(T,z)}}
WHILE z != T.root DO
    IF z.parent.parent == nil THEN
        zig(T,z);
    ELSE
        IF z == z.parent.parent.left.left OR
           z == z.parent.parent.right.right THEN
           zigZig(T,z);
        ELSE
            zigZag(T,z);
\end{codeBlock}
%\end{noindent}

\index{Zig-Zag} \includegraphics[width=12cm]{pictures/zigzag\IfDarkModeT{_dark}.PNG}
\clearpage
\index{Zig-Zig} \includegraphics[width=11cm]{pictures/zigzig\IfDarkModeT{_dark}.PNG}\\
\index{Zig} \includegraphics[width=11cm]{pictures/zig\IfDarkModeT{_dark}.PNG}\\

\vspace*{-2em}
\begin{description}[itemsep=.8em]
    \item[Suchen] \begin{itemize}
            \item Laufzeit: $O(h)$
            \item Suche des Knotens wie im BST
            \item Hochspülen des gefundenen Knotens (alternativ zuletzt besuchter Knoten, falls nicht gefunden)
        \end{itemize}
    \item[Einfügen] \begin{itemize}
            \item Laufzeit: $O(h)$
            \item Suche der Position wie im BST
            \item Einfügen und danach hochspülen des eingefügten Knotens
        \end{itemize}
    \item[Löschen] \begin{itemize}
            \item Laufzeit: $O(h)$
            \item[1.] Spüle gesuchten Knoten per Splay-Operation nach oben
            \item[2.] Lösche den gesuchten Knoten (Wenn einer der beiden entstehenden Teilbäume leer, dann fertig)
            \item[3.] Spüle den grö\ss{}ten Knoten im linken Teilbaum nach oben (kann kein rechtes Kind haben)
            \item[4.] Hänge rechten Teilbaum an grö\ss{}ten Knoten aus 3. an
        \end{itemize}
    \item[Laufzeit Splay-Bäume] \begin{itemize}
            \item Amortisierte Laufzeit: Laufzeit pro Operation über mehrere Operationen hinweg
            \item Worst-Case-Laufzeit pro Operation: $O(log_n~n)$
        \end{itemize}
\end{description}

\clearpage
\subsection{Binäre Max-Heaps}\label{Binaere Max-Heaps}\index{Binäre Max-Heaps}
\begin{definition}[Binäre Max-Heaps]\mbox{}
    \begin{itemize}
        \item Heaps sind keine BSTs
        \item Eigenschaften von binären Max-Heaps:
              \begin{itemize}
                  \item bis auf das unterste Level vollständig und dort von links gefüllt ist
                  \item Für alle Knoten gilt: \texttt{x.parent.key $\geq$ x.key}
                  \item Maximum des Heaps steht damit in der Wurzel
              \end{itemize}
        \item $h \leq log~n$, da Baum fast vollständig
    \end{itemize}
\end{definition}

\paragraph{Heaps durch Arrays}\mbox{}\\
\includegraphics[width=8cm]{pictures/heapArr\IfDarkModeT{_dark}.PNG}

\paragraph{Einfügen}\mbox{}
\begin{idea}[Einfügen]
    Einfügen und danach Vertauschen nach oben, bis Max-Eigenschaft wieder erfüllt ist
\end{idea}

Code:
%\begin{noindent}
\begin{codeBlock}[autogobble]{title={insert(H,k) // als unbeschränktes Array}}
H.length = H.length + 1;
H.A[H.length-1] = k;
i = H.length - 1;
WHILE i > 0 AND H.A[i] > H.A[i.parent]
    SWAP(H.A, i, i.parent);
    i = i.parent;
\end{codeBlock}
%\end{noindent}
Laufzeit: $O(h) = O(log~n)$
\clearpage
\paragraph{Lösche Maximum}
\begin{enumerate}
    \item[1.] Ersetze Maximum durch "`letztes"' Blatt
    \item[2.] Vertausche Knoten durch Maximum der beiden Kinder (\texttt{heapify})
\end{enumerate}
\index{extract-max(H)}\index{heapify(H,i)}
\begin{minipage}[t]{0.57\textwidth}\mbox{}
    %\begin{noindent}
    \begin{codeBlock}[autogobble]{title={extract-max(H)}}
    IF isEmpty(H) THEN return error "underflow";
    ELSE
        max = H.A[0];
        H.A[0] = H.A[H.length - 1];
        H.length = H.length - 1;
        heapify(H, 0);
        return max;
    \end{codeBlock}
    %\end{noindent}
\end{minipage}
\begin{minipage}[t]{0.43\textwidth}\mbox{}
    \centering
    \includegraphics[width=\textwidth]{pictures/heapDelete\IfDarkModeT{_dark}.PNG}
    \captionof{figure}{Beispiel Löschen des Maximums im Binären Max-Heap}
\end{minipage}

%\begin{noindent}
\begin{codeBlock}[autogobble]{title={heapify(H, i)}}
maxind = i;
IF i.left < H.length AND H.A[i]<H.A[i.left] THEN
    maxind = i.left;
IF i.right < H.length AND H.A[maxind]<H.A[i.right] THEN
    maxind = i.right;
IF maxind != i THEN
    SWAP(H.A, i, maxind);
    heapify(H, maxind);
\end{codeBlock}
%\end{noindent}

\paragraph{Heap-Konstruktion aus Array}
\begin{itemize}
    \item Blätter sind für sich triviale Max-Heaps
    \item Bauen von Max-Heaps für Teilbäume mithilfe Rekursion per \texttt{heapify}
    \item (Array nicht unbedingt in richtiger Reihenfolge)
\end{itemize}
\index{buildHeap(H,A)}\index{HeapSort(H,A)}
%\begin{noindent}
\begin{codeBlock}[autogobble]{title={buildHeap(H.A) // Array in H.A}}
    H.length = A.length;
    FOR i = ceil((H.length-1)/2) - 1 DOWNTO 0 DO
        heapify(H.A,i);
\end{codeBlock}
%\end{noindent}


\paragraph{Heap-Sort}
\begin{itemize}
    \item Idee: Bauen des Heaps aus Array und dann (wiederholte) Extraktion des Maximums
\end{itemize}
%\begin{noindent}
\begin{codeBlock}[autogobble]{title={heapSort(H.A)}}
buildHeap(H.A)                  // Bauen des Heaps
WHILE !isEmpty(H) DO
    PRINT extract-max(H);       // Ausgabe des Maximums bis Heap leer ist
\end{codeBlock}
%\end{noindent}

\clearpage
\subsection{B-Bäume}\label{B-Baeume}\index{B-Baeume}
\begin{definition}[B-Baum]\mbox{}
    \begin{itemize}
        \item Jeder B-Baum hat einen angebenen Grad also z.B. $t=2$
        \item Eigenschaften:
              \begin{itemize}
                  \item Wurzel zwischen $[1,~\cdots,~2t-1]$ Werte
                  \item Knoten zwischen $[t-1,~\cdots,~2t-1]$ Werte
                  \item Werte innerhalb eines Knotens aufsteigend geordnet
                  \item Blätter haben alle die gleiche Höhe
                  \item Jeder innere Knoten mit $n$ Werten hat $n+1$ Kinder, sodass gilt:
                  \item[] $k_0 \leq key[0] \leq k_1 \leq key[1] \leq ... \leq k_{n-1} \leq key[n-1] \leq k_n$
              \end{itemize}
    \end{itemize}
\end{definition}

\includegraphics[width=15cm]{pictures/bbaum\IfDarkModeT{_dark}.PNG}
\begin{grayInfoBox}
    \begin{itemize}
        \item Höhe B-Baum: $h \leq log_t \frac{n+1}{2}$ (Grad $t$ und $n$ Werte)
        \item B-Baum wird für grö\ss{}ere $t$ flacher
    \end{itemize}
\end{grayInfoBox}
\clearpage
\paragraph{Suche}\mbox{}\\
\includegraphics[width=12cm]{pictures/bbaumSuche\IfDarkModeT{_dark}.PNG}
%\begin{noindent}
\begin{codeBlock}[autogobble]{title={search(x, k)}}
WHILE x != nil DO
    i = 0;
    WHILE i < x.n AND x.key[i] < k DO
        i++;
    IF i < x.n AND x.key[i] == k THEN
        return(x, i);
    ELSE
        x = x.child[i];
return nil;
\end{codeBlock}
%\end{noindent}


\paragraph{Einfügen}
\begin{itemize}
    \item Einfügen erfolgt immer in einem Blatt
    \item Falls das Blatt voll ist, muss jedoch gesplittet werden
    \item $\Rightarrow$ Beim Durchlaufen des Baumes an jeder notwendigen vollen Position splitten
    \item Splitten:\index{Splitten}
          \begin{itemize}
              \item Bricht volle Node auf und fügt mittleren Wert zur Elternnode hinzu
              \item Aus den anderen Werten entstehen nun jeweils eigene Kinder
              \item An der Wurzel splitten erzeugt neue Wurzel und erhöht Baumhöhe um eins
          \end{itemize}
    \item Ablauf zusammengefasst:
          \begin{enumerate}
              \item Start bei Wurzel, falls kein Platz mehr splitten
              \item Durchlaufen des Baumes bis zur richtigen Position und immer, falls voll, splitten
              \item Einfügen der Node (fertig)
          \end{enumerate}
\end{itemize}
%\begin{noindent}
\begin{codeBlock}[autogobble]{title={insert(T, z)}}
Wenn Wurzel schon 2t-1 Werte hat, dann splitte Wurzel
Suche rekursiv Einfügeposition:
    Wenn zu besuchendes Kind 2t-1 Werte hat, splitte es erst
Füge z in Blatt ein
\end{codeBlock}
%\end{noindent}


\clearpage

\paragraph{Löschen}
\begin{itemize}
    \item Wenn Blatt noch mehr als $t-1$ Werte, kann der Wert einfach entfernt werden
    \item Allerdings durchlaufen wir hier den Baum auch wieder von oben und stellen gewisse Voraussetzungen her
    \item Durchlaufen des Baumes von oben und Anwendung der folgenden Algorithmen
\end{itemize}
\begin{minipage}{0.25\textwidth}
    \includegraphics[width=4cm]{pictures/bbaumMelt\IfDarkModeT{_dark}.PNG}
\end{minipage}
\begin{minipage}{0.75\textwidth}
    Allgemeines Verschmelzen:\index{Verschmelzen}
    \begin{itemize}
        \item Kind und rechter und linker Geschwisterknoten (sofern existent) nur $t-1$ Werte
        \item Wenn Elternknoten vorher min. $t$ Werte
        \item[] $\Rightarrow$ keine Änderung oberhalb notwendig
    \end{itemize}
\end{minipage} \vspace{1em}\\
\begin{minipage}{0.25\textwidth}
    \includegraphics[width=4cm]{pictures/bbaumRot\IfDarkModeT{_dark}.PNG}
\end{minipage}
\begin{minipage}{0.75\textwidth}
    Allgemeines Rotieren/Verschieben:
    \begin{itemize}
        \item Kind nur $t-1$ Werte
        \item Geschwister jedoch mehr als $t-1$ Werte
        \item keine Änderung oberhalb notwendig
    \end{itemize}
\end{minipage}

Code:

%\begin{noindent}
                    \begin{codeBlock}[autogobble]{title={delete(T, k)}}
                    Wenn Wurzel nur 1 Wert und beide Kinder t-1 Werte, 
                    verschmelze Wurzel und Kinder (reduziert Höhe um 1)
                    Suche rekursiv Löschposition:
                        Wenn zu besuchendes Kind nur t-1 Werte, 
                        verschmelze es oder rotiere/verschiebe
                    Entferne Wert k im inneren Knoten/Blatt             
                    // Ohne Probleme, aufgrund vorheriger Anpassung
                    \end{codeBlock}
                %\end{noindent}


\paragraph{Laufzeiten}
\begin{description}[leftmargin=2cm]
    \item [Einfügen] $\Theta(log_t~n)$
    \item [Löschen] $\Theta(log_t~n)$
    \item [Suchen] $\Theta(log_t~n)$
\end{description}
\begin{itemize}
    \item Nur vorteilhaft wenn Daten blockweise eingelesen werden
    \item $\mathcal{O}$-Notation versteckt hier konstanten Faktor $t$ für Suche innerhalb eines Knotens
\end{itemize}
\clearpage
\end{document}
