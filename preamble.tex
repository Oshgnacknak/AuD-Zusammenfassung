%We do not define Documentclass here
\usepackage[utf8]{inputenc}
\usepackage[T1]{fontenc}
\usepackage[ngerman]{babel}
%\usepackage{minted} % Used to syntax highlight code
\usepackage{tcolorbox}
\tcbuselibrary{minted,skins}
\usepackage{amsmath}
\usepackage{amssymb}
\usepackage{amsthm}
\usepackage{mathtools}
\usepackage{mleftright}
\usepackage{enumitem}
\usepackage{tikz}
\usetikzlibrary{patterns, shapes, intersections, arrows, math, decorations,decorations.pathreplacing, decorations.pathmorphing, positioning, calc, automata, chains,matrix, arrows.meta, , shadows.blur, shapes.symbols}
\usepackage{cryptocode}
\usepackage{booktabs}
\usepackage[font=small, labelfont=sc, position=bottom]{caption}
\usepackage[labelfont=normalfont, position=bottom]{subcaption}
\usepackage{etoolbox}
%\usepackage{lmodern}
\usepackage{adjustbox}
\usepackage{framed}
\usepackage{ragged2e}
\usepackage{ebproof}
\usepackage{proof}
\usepackage{tabularx}
\usepackage{latexsym}
\usepackage{colortbl}
\usepackage{forest}
\usepackage{longtable}
\usepackage[longtable]{multirow}
\usepackage{stackengine}
\usepackage{caption}
\usepackage{xifthen}
\usepackage{karnaughmap}
\usepackage{listings}
\usepackage{blindtext}
\usepackage{standalone}
\usepackage{wrapfig}
\usepackage{xr-hyper}
\usepackage{import}
%\usepackage{imakeidx}
% Define Glossary style
\ifthenelse{\isundefined{\printindex}}{}{
    \makeindex[
        columns=3,
        title=Stichwortverzeichnis, 
        intoc, 
        options=-s mystyle
        ]
}
\usepackage{filecontents} % Style für Stichwortverzeichnis nicht in seperatem File
\begin{filecontents}{mystyle.ist}
headings_flag  1 % wir benutzen Überschriften
heading_prefix "{\\large\\noindent\\bfseries " % und setzen sie fett und Groß
heading_suffix "\\hfil}\\nopagebreak\n"% und links, nach ihnen kein Seitenumbruch
delim_0 "\\dotfill" % Punktzeile zwischen Einträgen und Seitenzahlen (Ebene 0)
delim_1 "\\dotfill" % Punktzeile zwischen Einträgen und Seitenzahlen (Ebene 1)
delim_2 "\\dotfill" % Punktzeile zwischen Einträgen und Seitenzahlen (Ebene 2)
delim_r "--" % Trenner zwischen Start und Ende eines Seitenbereiches
suffix_2p "\\,f." % Suffix bei einem bereich aus 2 Seiten
suffix_3p "\\,ff." % Suffix bei einem bereich aus 3 Seiten
\end{filecontents}

% Make Links blue
\ifthenelse{\isundefined{\hypersetup}}{}{
    \hypersetup{
        %colorlinks=true,
        %linkcolor=blue,
    }
}

% Document definitions (please change here instead of in the document)
\newcommand{\Fach}{AuD}
%\newcommand{\contributors}{}

\newtcblisting{ccode}[2][]{
    listing engine=minted,
    colback=bashcodebg!30,
    colframe=black!70,
    listing only,
    %minted style=colorful,
    minted language=c,
    minted options={linenos=true,numbersep=3mm,texcl=true,#1},
    left=5mm,enhanced,
    overlay={\begin{tcbclipinterior}\fill[black!25] (frame.south west)
                rectangle ([xshift=5mm]frame.north west);\end{tcbclipinterior}},
    #2
}
\definecolor{bashcodebg}{rgb}{0.85,0.85,0.85}
%Formatierungen für Beispiele in diesem Dokument. Im Allgemeinen nicht notwendig!
\let\file\texttt
\let\code\texttt
\let\pck\textsf
\let\cls\textsf

%Color Definitions
\definecolor{tablegreen}{RGB}{207, 228, 174}
\definecolor{tablered}{RGB}{255, 191, 191}
\definecolor{tableyellow}{RGB}{255, 250, 193}
\definecolor{tableblue}{RGB}{107, 207, 246}
\definecolor{arrowgreen}{RGB}{0, 165, 79}
\definecolor{clight2}{RGB}{212, 237, 244}
\definecolor{codegreen}{rgb}{0,0.6,0}
\definecolor{codegray}{rgb}{0.5,0.5,0.5}
\definecolor{codepurple}{rgb}{0.58,0,0.82}
\definecolor{backcolour}{rgb}{0.95,0.95,0.92}

% Makros
\newcommand{\fatsf}[1]{\textsf{\textbf{#1}}}
\newcommand\xrowht[2][0]{\addstackgap[.5\dimexpr#2\relax]{\vphantom{#1}}}
\newcommand{\tikzmark}[2][-3pt]{\tikz[remember picture, overlay, baseline=-0.5ex]\node[#1](#2){};}

\tikzset{brace/.style={decorate, decoration={brace}},
    brace mirrored/.style={decorate, decoration={brace,mirror}},
}

\newcounter{brace}
\setcounter{brace}{0}
\newcommand{\drawbrace}[3][brace]{%
    \refstepcounter{brace}
    \tikz[remember picture, overlay]\draw[#1] (#2.center)--(#3.center)node[pos=0.5, name=brace-\thebrace]{};
}

\newcounter{arrow}
\setcounter{arrow}{0}
\newcommand{\drawcurvedarrow}[3][]{%
\refstepcounter{arrow}
\tikz[remember picture, overlay]\draw (#2.center)edge[#1]node[coordinate,pos=0.5, name=arrow-\thearrow]{}(#3.center);
}

% #1 options, #2 position, #3 text 
\newcommand{\annote}[3][]{%
    \tikz[remember picture, overlay]\node[#1] at (#2) {#3};
}

% Setzt die Nummerierung von Listen auf (a), (b), ...
\setlist[enumerate]{label=(\alph*)}

\newenvironment{cpenumerate}[1][]{\begin{enumerate}[nosep, #1]}{\end{enumerate}}

\newlength{\leftstackrelawd}
\newlength{\leftstackrelbwd}
\def\leftstackrel#1#2{\settowidth{\leftstackrelawd}%
    {${{}^{#1}}$}\settowidth{\leftstackrelbwd}{$#2$}%
    \addtolength{\leftstackrelawd}{-\leftstackrelbwd}%
    \leavevmode\ifthenelse{\lengthtest{\leftstackrelawd>0pt}}%
    {\kern-.5\leftstackrelawd}{}\mathrel{\mathop{#2}\limits^{#1}}}

\newcommand{\gegeben}{\fatsf{Gegeben: }}
\newcommand{\zuzeigen}{\fatsf{Zu Zeigen: }}
\newcommand{\zuberechnen}{\fatsf{Zu Berechnen: }}
\newcommand{\zubestimmen}{\fatsf{Zu Bestimmen: }}
\newcommand{\anzugeben}{\fatsf{Anzugeben: }}
\newcommand{\loesung}{\fatsf{Lösung: }}
\newcommand{\rechnung}{\fatsf{Rechnung: }}



% Center Captions
\captionsetup[figure]{justification=centering}
\captionsetup[subfigure]{justification=centering}
\captionsetup[table]{justification=centering}

% Remove unwanted space from tables
\aboverulesep = 0mm \belowrulesep = 0mm